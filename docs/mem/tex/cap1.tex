%%%%%%%%%%%%%%%%%%%%%%%%%%%%%%%%%%%%%%%%%%%%%%%%%%%%%%%%%%%%%%%%%%%%%%%%%%%%%
% Chapter 1: Motivaci�n y Objetivos 
%%%%%%%%%%%%%%%%%%%%%%%%%%%%%%%%%%%%%%%%%%%%%%%%%%%%%%%%%%%%%%%%%%%%%%%%%%%%%%%

Como finalidad de la asignatura de T�cnicas Experimentales del Grado en Matem�ticas, 
se debe saber aplicar conocimientos en el �rea de las Matem�ticas de una forma profesional 
y poseer las competencias que suelen demostrarse por medio de la elaboraci�n y defensa de 
argumentos y la resoluci�n de problemas.

%---------------------------------------------------------------------------------
\section{Objetivo principal}
\label{1:sec:1}
  
Los matem�ticos suelen encontrarse con el problema de integrar funciones que no est�n definidas 
de forma expl�cita. Se pueden utilizar m�todos gr�ficos, pero los m�todos num�ricos son mucho m�s 
precisos.

El objetivo principal de este proyecto es investigar el m�todo num�rico ``La Regla de Simpson'' y aplicarlo
sobre la integral de la funci�n $f(x)=x^{2}\ cos\ x$, en el intervalo $[1,3]$.
