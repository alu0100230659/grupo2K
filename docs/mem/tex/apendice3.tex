\section{Aproximacion y error de la regla de Simpson}
\label{Apendice3:error1}

\begin{center}
\begin{footnotesize}
\begin{verbatim}
#! /usr/bin/python
###################################################################################
# Fichero: uso_simpson.py
###################################################################################
#
# Contenido: Comprobacion del error cometido al aplicar la regla de Simpson
#            al calculo de la integral de  la funcion f(x)=x**2*cos(x), 
#            en el intervalo [a,b].
#
# Autor/es: Adrian R. Mendioroz Morales
#           Roberto C. Palenzuela Criado
#
###################################################################################
# Fecha de creacion: 06 de mayo de 2013 
###################################################################################
from simpson import regla_simpson
from math import *

F=(2*3*cos(3)+(3**2-2)*sin(3))-(2*1*cos(1)+(1**2-2)*sin(1))
print '\nValor real= ',F

aprox=regla_simpson(lambda x:(x**2*cos(x)),1,3)
print '\nAproximacion por la regla de Simpson: ',aprox
print '\t\t      Error absoluto: ', abs(F-aprox)
print '\t\t      Error relativo: ', abs(F-aprox)/abs(F)
print "\n"
\end{verbatim}
\end{footnotesize}
\end{center}


\section{Aproximacion y error de la regla de Simpson}
\label{Apendice3:error2}

\begin{center}
\begin{footnotesize}
\begin{verbatim}
#! /usr/bin/python
###################################################################################
# Fichero: uso_simpson_compuesta.py
###################################################################################
#
# Contenido: Comprobacion de los errores cometidos al aplicar la regla de Simpson
#            compuesta al calculo de la integral de la funcion f(x)=x**2*cos(x), 
#            en el intervalo [a,b].
#
# Autor/es: Adrian R. Mendioroz Morales
#           Roberto C. Palenzuela Criado
#
###################################################################################
# Fecha de creacion: 06 de mayo de 2013 
###################################################################################
from simpson import regla_simpson_compuesta
from math import *

F=(2*3*cos(3)+(3**2-2)*sin(3))-(2*1*cos(1)+(1**2-2)*sin(1))
print '\nValor real= ',F

print '\nAproximacion por la regla de Simpson compuesta: '
print '%25s %15s %15s %15s' % ('Numero de subintervalos','Aproximacion','Error absoluto',
'Error relativo')
n=[2,6,10,50,100]
aprox=0
i=0
for l in n:
  aprox=regla_simpson_compuesta(lambda x:(x**2*cos(x)),1,3,l)
  print '%16d %24.10f %15.10f %15.10f' %(n[i],aprox,abs(F-aprox),abs(F-aprox)/abs(F))
  i=i+1
print "\n"
\end{verbatim}
\end{footnotesize}
\end{center}